% !TEX root = ../main.tex

\begin{abstract}

When consumers trade financial products, they typically use well-identified service providers that operate under government regulation. In theory, decentralized platforms like Ethereum can offer trading services `on-chain' without an obvious entry point for regulators. Fortunately for regulators, most trading volume in blockchain-based assets is still on centralized service providers for performance reasons. However this leaves the following research questions we address in this paper: (i) is secure trading (\ie resistant to front-running and price manipulation) even feasible as a fully `on-chain' service on a public blockchain, (ii) what is its performance benchmark, and (iii) what is the performance impact of novel techniques (\eg `rollups') in closing the performance gap? 

To answer these questions, we `learn by doing' and custom design an Ethereum-based call market (or batch auction) exchange, \cm, with favourable security properties. We conduct a variety of optimizations and experiments to demonstrate that this technology cannot expect to exceed a few hundred trade executions per block (\ie ~13s window of time). However this can be scaled dramatically with off-chain execution that is not consumer-facing. We also illustrate, with numerous examples throughout the paper, how blockchain deployment is full of nuances that make it quite different from developing in better understood domains (\eg cloud-based web applications). 

\end{abstract}