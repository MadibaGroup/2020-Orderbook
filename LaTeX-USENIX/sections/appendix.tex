% !TEX root = ../main.tex

\section{Clearing storage (extended discussion)}

For space considerations, we kept the discussion in Section~\ref{sec:gasrefund} brief but the subject could use more explanation. First, we discuss the Ethereum's gas model--- (i) what are different gas parameters one needs to take into account when sending a transaction on the Ethereum blockchain, (ii) what is the gas refund, and (iii) and how it is calculated.  Next, we discuss one challenge in \cm, \textit{clearing mappings}, that turns out to be more difficult than we initially expected.

\subsection{Ethereum's Gas Model}

DApps written in high-level programming languages are compiled and translated into a compact representation (called ‘bytecode’) to be further executed on the Ethereum virtual machine (EVM). When a transaction is executed, each opcode in the execution path accrues a fixed, pre-specified amount of gas. The function caller will pledge to pay a certain amount of ETH (typically quoted in units of \textbf{Gwei}, where \textit{wei} is the smallest transactional unit of ETH) per gas, and miners are free to choose to execute that transaction or ignore it. The function caller is charged for exactly the amount the transaction costs and they cap the maximum they are willing to be charged (\textit{gas limit})---if the cap is too low to complete the execution, the miner keeps the Gwei and reverts the state of the DApp (as if the function never ran).

In Ethereum, transactions are bundled into blocks. A miner can include as many transactions (typically preferring transactions that bid the highest for gas) that can fit under a pre-specified \texttt{block gas limit}, which is algorithmically adjusted for every block. As of the time of writing, the limit is around 11M gas.

\subsubsection{Gas Refunds.} In order to reconstruct the current state of Ethereum's EVM, a node must obtain a copy of every variable change since the genesis block (or a more recent `checkpoint' that is universally agreed to). For this reason, stored variables persist for a long time and, at first glance, it seems pointless to free up variable storage (and unclear what `free up' even means). Once the current state of the EVM is established by a node, it can forget about every historical variable changes and only concern itself with the variables that have non-zero value (as a byte-string for non-integers) in the current state (uninitialized variables in Ethereum have the value zero by default). Therefore, freeing up variables will reduce the amount of state Ethereum nodes need to maintain going forward.

For this reason, some EVM operations cost a negative amount of gas. That is, the gas is refunded to the sender at the end of the transaction, however (1) the refund is capped at 50\% of the total gas cost of the transaction, and (2) the block gas limit applies to the pre-refunded amount (\ie a transaction receiving a full refund can cost up to 5.5M gas with a 11M limit). Negative gas operations include:

\begin{itemize}

\item \texttt{SELFDESTRUCT}. This operation destroys the contract that calls it and refunds its balance (if any) to a designated receiver address. \texttt{SELFDESTRUCT} operation does not remove the initial byte code of the contract from the chain. It always refunds 24,000 gas. For example, if a contract A stores a single non-zero integer and contract B stores 100 non-zero integers, the \texttt{SELFDESTRUCT} refund for both is the same (24,000 gas).


\item \texttt{SSTORE}. This operation loads a storage slot with a value. Using \texttt{SSTORE} to load a zero into a storage slot means the nodes can start ignoring it (recall that all variables, even if uninitialized, have zero by default). Doing this refunds 15,000 gas per slot. 

\end{itemize} 

At the time of this writing, Ethereum transaction receipts only account for the \texttt{gasUsed}, which is the total amount of gas units spent during a transaction, and users are not able to obtain the value of the EVM's refund counter from inside the EVM~\cite{signer2018gas}. So in order to account for refunds in Tables~\ref{tab:PQUnitTests}, we calculate them manually. First we need to figure out exactly how much storage is being cleared or how many smart contracts are being destroyed, then we multiply these numbers by 24,000 and 15,000 respectively. 

\subsection{Clearing Mappings}

Now let us put this theory into practice. In \cm, traders preload their account with ERC-20 tokens and/or ETH. \cm tracks what they are owed using a mapping called \texttt{totalBalance} and allows traders to withdraw their tokens at any time. However if a trader submits an order (\ie ask for their tokens), the tokens are committed and not available for withdrawal until the market closes (after which, the tokens are either transferred to the buyer or, if the trade is not executed, returned to the trader). Committed tokens are also tracked in a mapping called \texttt{unavailableBalance}. Sellers can request a token withdrawal up to their total balance subtracted by their unavailable balance.

As the DApp runs \texttt{closeMarket()}, it starts matching the best bids to the best asks. As orders execute, \texttt{totalBalance} and \texttt{unavailableBalance} are updated. At a certain point, the bids and asks will stop matching in price. At this point, every order left in the order book cannot execute (because the priority queue sorts orders by price, and so orders deeper in the queue have worst prices than the order at the head of the queue). Therefore all remaining entries in \texttt{unavailableBalance} can be cleared.

In Solidity, it is not possible to delete an entire mapping without individually zero-ing out each entry key-by-key. At the same time, it is wasteful to let an entire mapping sit in the EVM when it will never be referenced again. The following are some options for addressing this conflict.

\begin{enumerate}

\item \textbf{Manually Clearing the Mapping.} Since mappings cannot be iterated, a common design pattern used by DApp developers is to store keys in an array and iterate over the array to zero out each mapping and array entry. Clearing a mapping this way costs substantially more to clear than what is refunded (15,000 for each element).

\item \textbf{Store the mapping in a separate DApp.} We could wrap the mapping inside its own DApp and when we are done with the mapping, we can run \texttt{SELFDESTRUCT} on the contract. This refunds us 24,000 gas which is less than the cost of deploying the extra contract. Additionally, every call to the mapping is more expensive because (1) it is an external function call, and (2) function calls  to be evaluated to ensure they only come from the market contract (if a mapping is a local variable, you get private access for free). 

\item \textbf{Leave and Ignore the Mapping.} The final option is to not clear the mapping and just create a new one (or create a new prefix for all mapping keys to reflect the new version of the mapping). Unfortunately, this is the most economical option for DApp developers even if it is the worst option for Ethereum nodes. 

\end{enumerate}

Clearing storage is important for reducing EVM bloat. The Ethereum refund model should be considered further by Ethereum developers to better incentivize developers to be less wasteful in using storage. 



%% = = = = = = = = = = = =Transferring the Collateral back to the traders = = = = = = = = = = = = =  %
%\subsection{Collateralization}
%
%Options for non-collateralized markets
%ERC20 to ERC20 -> extra transaction
%



