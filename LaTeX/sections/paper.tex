% !TEX root = ../main.tex



% = = = = = = = = = = = = = = = = = = = = = = = = = = = = = = = = = = = = = = = = = =
\section{Introductory Remarks}

Blockchain is a type of distributed database (or ledger) that is open to anyone who wants to participate, robust against a wide range of faulty and malicious behaviours, and runs without anyone in charge. When a participant looks at her local copy of the ledger, she is assured that (i) anyone has the exact same records and (ii) each record was validated by the "majority" (technically, a computational majority) of participants before it was written into the ledger. Blockchain technology has rapidly gained global interest and become one of the most significant technologies in recent years. It was first introduced in 2008 by Satoshi Nakamoto as an underlying technology of Bitcoin, a peer to peer electronic cash system~\cite{nakamoto2019bitcoin}, whose currency reached a market capitalization of \$137 billion as of January 2020. In 2014, a new blockchain based application known as Ethereum was introduced by Buterin~\cite{buterin2014next}. By implementing a decentralized virtual machine, Ethereum virtual machine (EVM), Ethereum allows network users to execute programmable smart contracts on it. So developers are now able to build decentralized applications (DApps) that are executed correctly according to the consensus rules of the Ethereum network. 

% = = = = = = = = = = = = = = = = =
\subsection*{Contributions.}





% = = = = = = = = = = = = = = = = = = = = = = = = = = = = = = = = = = = = = = = = = =
\section{Preliminaries}

Strong interest in blockchain and distributed ledger technologies has led into the high trading volume of digital assets and emergence of fast growing cryptocurrency exchanges all over the world. These crypto exchanges are similar to stock exchanges but, as the name suggests, facilitate the trading process of different cryptocurrency tokens. Basically, such exchanges provide a trading venue where market participants can buy and/or sell cryptocurrency assets among one another. In this section, we describe two types of crypto exchanges.


\subsection{Blockchain}



\begin{itemize}

% transparency, front-running:
\item \textbf{Transparency and Front-running.} Blockchains (in general)  are transparent --- transactions are first relay around the network before being selected by a miner, validated, and included in a block. This allows malicious nodes/miners to observe and front-run the transactions by dropping them or attempting to insert their own transactions before the observed transactions~\cite{eskandari2019sok}. In the context of a DEX, front-running attacks enable malicious nodes to have their orders executed before others and benefit from it.


% slow but is becoming more efficient over time, expensive
\item \textbf{Speed.} Blockchains are not fast architectures --- speed is one of the factors that diminishes the real world potential of blockchains. Blockchains are slow-moving as it takes block interval (\ie time it takes to mine a block) for transactions to be confirmed and included in valid blocks. So it is difficult to push a stock market on the blockchain as stock markets move extremely fast. High-frequency trading (HFT) is a trading method example that uses active computers to process a large number of trades in fractions of a second. In Ethereum 1.0 blocks are produced at intervals every 10 to19 seconds, note that this is not yet suitable for time sensitive transactions. However, capabilities of blockchains will be further enhanced ad they will become more efficient over time; Ethereum 2.0 has a shorter block time of 12 seconds. 

% Gas model: gas limit, how much, per transaction vs per block, Solidity/EVM quirks: gas refunds
\item \textbf{Gas Parameter in Ethereum Transactions.} 

Gas is the Ethereum's pseudo-currency and a key variable for the execution of smart contracts. DApps written in high-level programming languages are compiled and translated into a compact representation (called ‘bytecode’) to be further executed on the Ethereum virtual machine or EVM. Each of these opcodes has a fixed amount of gas assigned and is a measure of computational effort. Separating ETH from gas prevents transactions cost from being too expensive as they are less impacted by ETH price fluctuations. Gas unit does not a have a monetary value and as mentioned, it only  measures the computational work undertaken by miners. To pay miners, Ethereum introduces a \texttt{gas price} --- small denomination of ether called \textbf{Gwei} attached to each gas unit. Essentially, gas price indicates how much users are willing to pay per unit of gas, clearly the higher gas price leads to the faster execution of a transaction. Miners also need to know the total amount of computational work a user is requesting, called the \texttt{gas limit}. \texttt{Gas limit} is a parameter that limits the amount of gas users would spend in a transaction. It protects users from spending unlimited ETH on their transactions and must be set carefully. If the \texttt{gas limit} is too low, the transaction will exceed the limit, all operations will be reverted while the user must pay for the computational work performed by the miner. By default the gas limit for an Ethereum transaction is 21,000 gas. An Ethereum block also has a \texttt{block gas limit} field which is set by the Ethereum miners and indicates the maximum amount of gas all the transactions in that block are allowed to consume, the Ethereum \texttt{block gas limit} is currently 11,741,495. \footnote{\[July 2020:\] https://ethstats.net/} 


\subsection*{Gas Refunds.}There are two particular EVM operations with negative gas --- certain amount of gas is refunded to the sender at the end of the transaction. These operation include:

\paragraph{\texttt{SELFDESTRUCT}.}This operation destroys and deletes the originating contract and refunds its balance (if any) to a designated receiver address. Note that the Ethereum storage is implemented in the form of a has map and EVM is not statistically aware of which storage slots are held by the contract. Because of this the \texttt{SELFDESTRUCT} operation does not remove the initial byte code of the contract from the chain, but it frees up the state storage and has a refund of 24,000 gas.

\paragraph{\texttt{SSTORE}.} This operation clears the Ethereum storage and has15,000 gas refunds. 


Note that in order to urge miners to process smart contracts with refunds, the accumulated gas refund can never exceed half the gas used up during computation~\cite{wood2014ethereum}. So at the end of a successful transaction, the amount of gas in the refund counter (capped at half the net gas used) is returned to the caller. At the time of this writing, Ethereum transaction receipts only account for the \texttt{gasUsed}, which is the total amount of gas units spent during a transaction, and we are not able to obtain the value of the EVM's refund counter from inside the EVM~\cite{signer2018gas}. So in order to account for refunds, we decide to calculate them manually; first we figure out exactly how much storage is being cleared or how many smart contracts are being destroyed, then we multiply these numbers by 24,000 and 15,000 respectively. 




%gas price: how much you pay per gas unit
%gas limit: how much work you are requesting 
%gas cost (Ether) = gasPrice * gasCost(gas)
%gas limit: is a parameter that limits the amount of gas you spend in a transaction (how to estimate? using estimateGas)
%gas price how to estimate: using ethgassation 
%by defaluth the gas minimum gas limit for ethereum txs is 21000 gas
%a block also has a gas limit field, it defines a maximum amount of gas all txs in th eblock combined are allowed to consume. this block gas limit determines the maximum number of transactions within the block. this block gas limt is not fixed and set by miners  
%
%ers to process smart contracts with refunds, the accumulated gas refund can never exceed half the gas used up during computation~\cite{wood2014ethereum}. So at the end of a successful transaction, the amount of gas in the refund counter (capped at half the net gas used) is returned to the caller. 


Based on these issues blockchains cannot solve this particular problem. Many stock exchanges that currently operate on blockchains tackle this issue by using a private network (where not every network participant can be a node or miner) together with a central time-server that establishes the time priority. We propose a different method of implementing a fully decentralized exchange using an alternative data structure known as \textit{a call market} or \textit{frequent batch auction} instead of a time-sensitive order book~\cite{clark2014decentralizing}.

\end{itemize}

\subsection{Exchanges}

% = = = = = = = = = = = = = = = = =
\subsubsection{Centralized Exchanges (CEX)}\label{sec:CEX}

This group of exchanges operate in a traditional manner by providing a centralized platform controlled by exchange operators. Essentially, the exchange acts as a trusted party whom the market participants must trust to handle their assets. Traders transfer the ownership of their assets to the exchange who safeguards customer funds using different methods. Centralized exchanges ease trading process of digital assets as they do not place the burden of safeguarding of the assets onto individualst, however, traders must fully trust the exchange operator in taking custody of their assets (\eg Coinbase).~\footnote{https://www.coinbase.com} Similar to any centralized platform these exchanges are not completely immune to malicious activities. For example in February 2014, a famous centralized Tokyo-based Bitcoin exchange called MtGox was hacked which led to loss of 650,000 Bitcoins~\cite{TheHisto45:online}. Also, a malicious exchange operator can simply steal users' funds; in February 2019, QuadrigaCX, Canada’s well-known centralized cryptocurrency exchange, claimed that it cannot access to \$190-million worth of customers’ funds~\cite{SEBIOrde83:online}.


% = = = = = = = = = = = = = = = = =
\subsubsection{Decentralized Exchanges (DEX)}

Unlike centralized exchanges, DEXes facilitate a secure automated peer-to-peer trading process on blockchains (\eg Ethereum) for market participants with no third parties involved, hence solving the issues that are inherent in centralized exchanges. This group of exchanges provide a better security and privacy by enabling traders to remain fully authoritative over their funds and personal data. With rising of Decentralized Finance (DeFi) or Open Finance movements, the number of DEX platforms on Ethereum have recently exploded and large volume of trading now occurs on DeFi exchanges (\eg Uniswap).~\footnote{https://uniswap.org}


However due to the high cost and technical limitations introduced by blockchain technologies, building a fully decentralized exchange has remained a major challenge. In the following section, we explore the idea of building a decentralized exchange on Ethereum.


% = = = = = = = = = = = = = = = = = = = = = = = = = = = = = = = = = = = = = = = = = =

\subsection{Order Books}

Order books are electronic data structures that maintain lists of bid and ask orders for various assets (\eg currencies, stocks, bonds \etc) in specific markets. The most common version of order books is a \textit{double auction}, where market participants submit their bid and ask orders and the market clearing price will be calculated as the average between the best bid and ask prices. Order books often sort orders based on their price and submission time, this order allocation technique is called price-time priority~\cite{preis2011price} where orders are prioritized from highest price to lowest and given any two orders with the same price, they will be sorted based on their submission timestamps.

Looking more closely, order books are rather electronic ledgers that get updated over time. Given the definition of the Ethereum blockchain, that is a distributed ledger, the idea of implementing an order book in the form of a smart contract will seemingly resolve the existing issues with centralized exchanges (see Section~\ref{sec:CEX}). However, this design is not feasible due to some technical limitations that exist within blockchains:

% = = = = = = = = = = = = = = = = =
\begin{itemize}
%---------+++++++++++++++++++-----------------

\item \textbf{Speed.} All blockchains are slow-moving as it takes block intervals for transactions to get confirmed. On the other hand stock markets move very fast;  high-frequency trading (HFT) is a trading method example that uses active computers to process a large number of trades in fractions of a second. 

%---------+++++++++++++++++++-----------------

\item \textbf{Front-running and Censorship.} Traders must broadcast their order messages to the network before they can be placed in the block (\ie order book). In such scenario, privileged nodes and miners can front-run transactions or completely drop (censor) those that are competing with their own~\cite{eskandari2019sok}. Also, miners could almost control how the order book gets updated by inserting their own order messages after viewing all other orders and before mining the block .

%---------+++++++++++++++++++-----------------

\item \textbf{Enforcing Time.} There is no notion of time on blockchains.I if two orders arrive at the same price, in the normal world whoever is the first wins. However, we cannot enforce the priority of orders in the blockchain space; if Alice is more well-connected to the network her order gets broadcast better and executed first although she sends it after Bob. 

%---------+++++++++++++++++++-----------------
\end{itemize}
% = = = = = = = = = = = = = = = = =


% = = = = = = = = = = = = = = = = = = = = = = = = = = = = = = = = = = = = = = = = = =

\section{Priority Queue}\label{sec:pq}

% = = = = = = = = = = = =Operations for a generic Priority Queue Table = = =  = = = = = = = = = = = %
\begin{table}[t]
\centering
\begin{tabular}{|c|c|}
\hline

\textbf{Operation}   & \textbf{Description}    \\ \hline

% = = = = = = = = = = = = = = = = = = = = = = = = = = = = = = = = = = = = = = = = = = = = = = = = %
	\textbf{Enqueue()}       	& inserts an element into the priority queue                        \\ \hline
	\textbf{Dequeue()}		& removes and returns the highest priority element 		\\ \hline
	\textbf{isEmpty()}			& checks if the priority queue is empty 					\\ \hline
% = = = = = = = = = = = = = = = = = = = = = = = = = = = = = = = = = = = = = = = = = = = = = = = =  %

\end{tabular}
\caption{\footnotesize{Operations for a generic Priority Queue.}
\label{tab:PQ_API}}
\end{table}
% = = = = = = = = = = = = = = = = = = = = =  = = = = = = = = = = =  = = = = = = = = = == = = =  = =%

In designing \cm, performance within Ethereum's limited gas model became the main bottleneck. Within a call market, recording trades and accessing them to close the market is the most time consuming step. For this reason, we decided to do a deep study of the best data-structure for this specific task. While data structures are well studied for many languages, Solidity/EVM have their own quirks (\eg gas refunds, a relatively cheap mapping data structure, and no actual support for object oriented programming) that make them difficult to assess which will perform best without actually deploying and evaluating each variant. 

When closing a call market, the orders are examined in order: highest to lowest price for bids, and lowest to highest price for asks. It most circumstances, the market closing algorithm does not have to consider any deeper bids/asks from the list when choosing whether the current best bid and ask can be fulfilled. The only exception is in the case of a tie on price, which we return to later. For this reason, the ideal data structure for storing bids/asks is a \textit{priority queue} (see Table~\ref{tab:PQ_API}) where each order's priority is its price. Specifically, we use two PQs---one for bids where the highest price is the highest priority, and one for asks where the lowest price is the highest priority. 

There are numerous ways of implementing a PQ. A PQ has an underlying list---common options include a static array, dynamic array, and linked list. The most expensive operation is keeping the data sorted---common options include (i) sorting during each enqueue, (ii) sorting for each dequeue, or (iii) splitting the difference by using a heap as the underlying data structure. Respectively, the time complexities are (i) linear enqueue and constant dequeue, (ii) constant enqueue and linear dequeue, and (iii) logarithmic enqueue and logarithmic dequeue. This said, Solidity/EVM has its own specific considerations to take into account. As closing the market is very expensive with any PQ, we rule out using (ii) as fully sorting while dequeuing would be prohibitive. We experiment with the following 5 options for (i) and (iii):

% = = = = = = = = = = = = = = = = =
\begin{enumerate}

%---------+++++++++++++++++++-----------------

\item\textbf{{Heap with Dynamic Array.}} A heap is a type of binary tree data structure that comes in two forms of a (i) Max-Heap and (ii) Min-Heap. All the nodes of tree are in a specific order and the root of the tree always represents the highest priority item of the data structure (the largest and smallest values in the Max-Heap and Min-Heap respectively). We implemented a priority queue with a heap that stores its data in a dynamically-sized array. 

%---------+++++++++++++++++++-----------------

\item\textbf{{Heap with Static Array.}} A heap can be also represented by a Solidity storage array in which the storage is statically allocated. To do this, we pass the required size of the array as a constructor parameter to the priority queue smart contract. 

%---------+++++++++++++++++++-----------------

\item\textbf{{Heap with Mapping.}} In the above implementations, the heap stores the entire order (as a struct) in the heap. In this variant, we store the order struct in a Solidity mapping and store the mapping's key in the heap.  

%---------+++++++++++++++++++-----------------

\item\textbf{{Linked List.}} In this variant, we insert a new element into its correct position (based on its price) when running enqueue. The PQ itself stores elements in a linked list (enabling us to efficiently insert a new element between two existing elements). Solidity is described as object-oriented but the equivalent of an object is an entire smart contract. Therefore an object-oriented linked list must either (i) create each node in the list as a struct---but this is not possible as Solidity does not support recursive structs---or (ii) make every node in the list its own contract. The latter option seems wasteful and unusual, but we try it out anyways. Thus each node is its own contract and contains the order data and a pointer to the address of the next contract in the list.

%---------+++++++++++++++++++-----------------

\item\textbf{{Linked List with Mapping.}} Finally, we try a variant of a linked list using a Solidity mapping. The value of the mapping is a struct with order data and the key of the next (and previous) node in the list. The contract stores the key of the first node (head) in the list. 

%---------+++++++++++++++++++-----------------

\end{enumerate}
% = = = = = = = = = = = = = = = = =
% = = = = = = = = = = = = = = = = = = = = = = = = = = = = = = = = = = = = = = = = = =

\subsection{Priority Queue Evaluation} 

\subsubsection{Enqueue performance.} We implemented, deployed, and tested each priority queue using Truffle and Ganache.\footnote{Github: Link removed for anonymity.} We tried a variety of tests (including testing the full call market with each varient) with consistent results in performance. A simple test to showcase the performance profile is shown in Figure~\ref{fig:random_insertion}. We simply enqueue 50 integers chosen at random from a fixed interval in each PQ variant. The bigger the PQ gets, the longer enqueue takes---a linear increase for the linked list variants, and logarithmic for the heap variants. 

% = = = = = = = = = = = =random_insertion Figure = = = = = = = = = = = = =  %

\begin{figure}[t]
\centering
\includegraphics[width=0.8\textwidth]{fig/random_insertion.png}
\caption{\footnotesize{Gas costs for enqueuing random items into five different priority queue variants. For the x-axis, a value of 10 indicates it is the 10th number entered in the priority queue. The y-axis is the cost of enqueue in gas.}  \label{fig:random_insertion}}
\end{figure}

% !TEX root = ../main.tex

% = = = = = = = = = = = =Gas costs and refunds for Dequeueing 50 integers from each PQ  = = = = = = = = = = = = =  %

\begin{table}[]
\centering
\begin{tabular}{|l|r|r|r|}

\multicolumn{1}{c}{\.} & \headrow{Gas Costs (\texttt{gasUsed})} & \headrow{Refund (Manual)} & \headrow{Full Refund?} \\ \hline

% = = = = = = = = = = = = = = = = = = = = = = = = = = = = = = = = = = = = = = = = = = = = = = = = = = = = = = = = = = = = = = = = = = = = = = %
Heap with Dynamic Array       & 2,518,131          & 750,000     &\full                  \\ \hline
% = = = = = = = = = = = = = = = = = = = = = = = = = = = = = = = = = = = = = = = = = = = = = = = = = = = = = = = = = = = = = = = = = = = = = = %
Heap with Static Array          	& 1,385,307                             & 750,000      &\full                \\ \hline
% = = = = = = = = = = = = = = = = = = = = = = = = = = = = = = = = = = = = = = = = = = = = = = = = = = = = = = = = = = = = = = = = = = = = = = %
Heap with Mapping 		& 2,781,684                            & 1,500,000    &\full                 \\ \hline
% = = = = = = = = = = = = = = = = = = = = = = = = = = = = = = = = = = = = = = = = = = = = = = = = = = = = = = = = = = = = = = = = = = = = = = %
Linked List                     	& 557,085               	           & 1,200,000      &\full                \\ \hline
% = = = = = = = = = = = = = = = = = = = = = = = = = = = = = = = = = = = = = = = = = = = = = = = = = = = = = = = = = = = = = = = = = = = = = = %
Linked List with Mapping      	& 731,514              	     	  &  3,765,000      &\full                 \\ \hline
% = = = = = = = = = = = = = = = = = = = = = = = = = = = = = = = = = = = = = = = = = = = = = = = = = = = = = = = = = = = = = = = = = = = = = = %

\end{tabular}
\caption{\footnotesize{The gas metrics associated with dequeuing 50 items from five priority queues variants. For the refund, (\full) indicates the  refund was capped at the maximum amount and (\prt) means a greater refund would be possible.}
\label{tab:PQUnitTests}}
\end{table}
% = = = = = = = = = = = = = = = = = = = = = = = = = = = = = = = = = = = = =  %


% = = = = = = = = = = = = = = = = = = = = = = = = = = = = = = = = = = = = = = = =  %


\subsubsection{Dequeue performance.} For each PQ variant storing 50 random integers, the \texttt{Dequeue()} function is iterated until the data structure is empty. The total gas cost for fully dequeuing the priority queue variants is outlined in Table~\ref{tab:PQUnitTests}. These tests are performed using the following Ethereum gas metrics: block gas limit $=$11,741,495 and 1 gas $=$ 56 gwei.\footnote{\[July 2020:\] https://ethstats.net/} Dequeuing removes data from the contract's storage. Recall this results in a gas refund. Based on our manual estimates (EVM does not expose the refund counter), every variant receives the maximum gas refund possible (\ie half the total cost of the transaction). In other words, each of them actually consumes twice the \texttt{gasUsed} amount in gas before the refund, however none of them are better or worst based on how much of a refund they generate.  

\subsubsection{Decision.} Based on enqueuing, the heap variants are the cheapest in terms of gas, while based on dequeuing, the link list variants are cheapest. This is in accordance with the theoretical worst-case time complexity for each. However, (i) the linked list variants are materially cheaper than the heap variants at dequeuing, and (ii) dequeuing in a call market must be done as a batch, whereas enqueuing is paid for one at a time by the trader submitting the order, and (iii) Ethereum will not permit more than hundreds of orders so the asymptotic behaviour is not a significant factor. For these reasons, we suggest using a linked list variant for this specific application. As it can be seen in Figure~\ref{fig:random_insertion}, the associated cost for inserting elements into a linked list PQ is significantly greater than the linked list with mapping, as each insertion cause the creation of a new contracts. Accordingly, we choose to implement the call market with the linked list with mapping. Overall this PQ balances a moderate gas cost for insertion (\ie order submission) with one for removal (\ie matching the orders). 


% = = = = = = = = = = = = = = = = = = = = = = = = = = = = = = = = = = = = = = = =  %
%  = = = = = = Table 2 (dequeue) Conversions = = = = = = %
%
%As of July 2020,  $1 gas = 56\times10^{-9} ether$\footnote{https://ethstats.net/} and 1 ether = \$238.33\footnote{https://coinmarketcap.com/}.
%
%$Transaction cost (eth) = GasUsed (unit) \times Gas Price (eth)$ 
%
%%  = = = = = = Transaction Costs in eth = = = = = = %
%
%Heap with Dynamic Array:    		$2,518,131 \times 56\times10^{-9} = 141,015,336\times 10^{-9}$
%Heap with Dynamic Array:     		$1,385,307 \times 56\times10^{-9} = 77,577,192\times 10^{-9}$
%Heap with Mapping:              		$2,781,684 \times 56\times10^{-9} = 155,774,304\times 10^{-9}$
%Linked List:						$557,085 \times 56\times10^{-9} = 	31,196,760\times 10^{-9}$
%Linked List with Mapping:			$731,514 \times 56\times10^{-9} = 	40,964,784\times 10^{-9}$
%
%%  = = = = = = Transaction Costs in USD = = = = = = %
%
%Heap with Dynamic Array:    		$141,015,336\times 10^{-9} \times 238.33 = \$33.60$
%Heap with Dynamic Array:     		$77,577,192\times 10^{-9} \times 238.33 = \$18.48$
%Heap with Mapping:              		$155,774,304\times 10^{-9} \times 238.33 = \$37.12$
%Linked List:						$31,196,760\times 10^{-9} \times 238.33 = \$7.43$
%Linked List with Mapping:			$40,964,784\times 10^{-9} \times 238.33 = \$9.76$
%
% = = = = = = = = = = = = = = = = = = = = = = = = = = = = = = = = = = = = =  %

\subsection{Cost/Benefit of cleaning up after yourself}
\label{sec:LLvsLLMapping}
\label{sec:gasrefund}

% !TEX root = ../main.tex


% = = = = = = = = = = = =Gas costs and refunds for Dequeueing 50 integers from each PQ  = = = = = = = = = = = = =  %

\begin{table}[t]
\centering
\begin{tabular}{|p{7cm}|r|r|r|}

\multicolumn{1}{c}{\.} & \headrow{Gas Costs (\texttt{gasUsed})} & \headrow{Refund (Manual)} & \headrow{Full Refund?} \\ \hline

% = = = = = = = = = = = = = = = = = = = = = = = = = = = = = = = = = = = = = = = = = = = = = = = = = = = = = = = = = = = = = = = = = = = = = = %
% = = = = = = = = = = = = = = = = = = = = = = = = = = = = = = = = = = = = = = = = = = = = = = = = = = = = = = = = = = = = = = = = = = = = = = %
Linked List w/o \texttt{SELFDESTRUCT}         		& 721,370          & 0     &$\times$  \\ \hline
Linked List with \texttt{SELFDESTRUCT}       		& 557,085          & 1,200,000     &\full  \\ \hline
Linked List with Mapping and w/o \texttt{DELETE}    & 334,689          & 765,000     &\full  \\ \hline
Linked List with Mapping and \texttt{DELETE}		& 731,514          & 3,765,000     &\full  \\ \hline

% = = = = = = = = = = = = = = = = = = = = = = = = = = = = = = = = = = = = = = = = = = = = = = = = = = = = = = = = = = = = = = = = = = = = = = %
% = = = = = = = = = = = = = = = = = = = = = = = = = = = = = = = = = = = = = = = = = = = = = = = = = = = = = = = = = = = = = = = = = = = = = = %
% = = = = = = = = = = = = = = = = = = = = = = = = = = = = = = = = = = = = = = = = = = = = = = = = = = = = = = = = = = = = = = = = = = = = = = %

\end{tabular}
\caption{The gas metrics associated with dequeuing 50 items from five priority queues variants. For the refund, (\full) indicates the  refund was capped at the maximum amount and (\prt) means a greater refund would be possible.\label{tab:cleaning}}
\end{table}


%---------+++++++++++++++++++-----------------
% = = = = = = = = = = = = = = = = = = = = = = = = = = = = = = = = = = = = = = = =  %



%  = = = = = = Table 3 (cleaning.tex) Conversions = = = = = = %

%As of July 2020,  $1 gas = 56\times10^{-9} ether$\footnote{https://ethstats.net/} and 1 ether = \$238.33\footnote{https://coinmarketcap.com/}.
%
%$Transaction cost (eth) = GasUsed (unit) \times Gas Price (eth)$ 
%
%%  = = = = = = Transaction Costs in eth = = = = = = %
%
%Linked List with \texttt{SELFDESTRUCT}:     				$557,085 \times 56\times10^{-9} =  31,196,760\times 10^{-9}$  
%Linked List w/o \texttt{SELFDESTRUCT}:     				$721,370 \times 56\times10^{-9} =  40,396,720\times 10^{-9}$ 
%Linked List with Mapping and \texttt{DELETE}:    			$731,514 \times 56\times10^{-9} =  40,964,784\times 10^{-9}$ 
%Linked List with Mapping and w/o \texttt{DELETE}: 		$334,689 \times 56\times10^{-9} =  18,742,584\times 10^{-9}$ 
%
%%  = = = = = = Transaction Costs in USD = = = = = = %
%
%Linked List with \texttt{SELFDESTRUCT}:     				$31,196,760\times 10^{-9}  \times 238.33 = \$7.43$
%Linked List w/o \texttt{SELFDESTRUCT}:     				$40,396,720\times 10^{-9} \times 238.33 = \$9.62$
%Linked List with Mapping and \texttt{DELETE}:    			$40,964,784\times 10^{-9}  \times 238.33 = \$9.76$
%Linked List with Mapping and w/o \texttt{DELETE}: 		$18,742,584\times 10^{-9} \times 238.33 = \$4.46$
% = = = = = = = = = = = = = = = = = = = = = = = = = = = = = = = = = = = = =  %



% With 50 selfdestructs

\subsubsection*{Linked List.} Removing the highest priority element from the list can be done in two ways. We can use the \texttt{SELFDESTRUCT} operation to destroy the head node of the list (\ie smart contract) from the Ethereum blockchain and transfer its funds (if any) to a payable address (\eg the current block miner or the call market contract). Although the \texttt{SELFDESTRUCT} operation does not remove the initial byte code of the contract from the chain, but it frees up the state storage and refunds 24000 gas to the caller. Another design perspective is to only update the head pointer of the linked list and leave the previous head (\ie smart contract) on Ethereum blockchain when removing the highest priority element from the list. Although the first technique is a bit more costly (because it adds an extra operation \texttt{SELFDESTRUCT} to the \texttt{dequeue()} function), it results in the less final net cost due to the 24000 gas refunds. Also, it is best to avoid storing unnecessary storage variables on chain. Table~\ref{tab:LLvsLLMapping} outlines different gas metrics for iteratively dequeuing highest priority elements in the two implementations of sorted linked lists that contain 50 unsigned integers.

%---------+++++++++++++++++++-----------------

\subsubsection*{Linked List with Mapping.} Similarly, linked list with mapping can be implemented in two ways. For dequeuing items from this priority queue we can use the \texttt{DELETE} method and remove the mapping element that contains the highest priority item from storage completely  \textbf{and} update the head and tail pointers of the list. Deleting a storage variable is identical to setting a non-zero variable to zero (\texttt{SSTORE} 0) that costs 20,000 gas but with 15,000 refunded~\cite{wood2014ethereum}. So although the \texttt{DELETE} method refunds in 15000 gas, it is yet a gas-costly operation  as it increases the overall gas cost of the transaction. In another technique, we can only update the head and tail pointers of the linked list when dequeueing the highest priority item and not delete the mapping element associated with the previous pointers. As it can be seen in the last two columns of Table~\ref{tab:LLvsLLMapping}, this scheme consumes less gas for dequeuing 50 unsigned integers compared to the previous technique where we clear up the chain, however, leaving unnecessary storage data on chain is not a proper design practice.


 that uses the  the \texttt{DELETE} method to clear up the chain every time the highest priority element is removed from the list. In 



% = = = = = = = = = = = = = Experimentations = = = = = = = = = = = = = = = = = = =  %

\section{Call Market Design}



In this research work, we design and implement a call market on the Ethereum blockchain. We divide the trading day into frequent but discrete time intervals; every time the market opens traders submit their ask and bid order messages. These orders are accumulated in a sorted data structure and when the trading period is over, orders are matched against each other. Our proposed decentralized exchange is both data structure and token agnostic (traders can trade ERC20 standard compliant token). The upcoming orders are not stored in the call market contract, instead they are stored in a priority queue where each element is associated with a priority and is served according to its priority.
 
As mentioned in~\ref{sec:callmarket}, Our system treats the time as discrete and not continuous; upcoming orders are accumulated over predetermined time intervals and are processed in batch instead of serially in order of receipt. The trading day is divided into frequent but discrete time intervals; every time the market opens traders submit their ask and bid order messages. These orders are accumulated in a linked list with mapping priority queue and when the trading period is over, orders are matched against each other. The call market smart contract is written in 268 lines of Solidity, a high level programming language that is syntactically similar to Java~\cite{Ethereum41:online}, using the Truffle development framework and deployed on Ganache-CLI, Table~\ref{tab:callmarket_functions} represents the call market's primary operations.

%GitHub repo:  \footnote{\CallMarketRepo}
% = = = = = = = = = = = = = = = = = = = = =Call market operations Table = = = = = = = = = = = =  = = = = = =  %
\begin{table}[]
\begin{tabular}{|l|l|}
\hline
\multicolumn{1}{|c|}{\textbf{Operation}} & \multicolumn{1}{c|}{\textbf{Description}}                            			\\ \hline

% = = = = = = = = = = = = = = = = = = = = = = = = = = = = = = = = = = = = = = = = = = = = = = = = = = = =  %
	DepositToken()                           & Deposits ERC20 standard compliant tokens in the call market contract \\ \hline
% = = = = = = = = = = = = = = = = = = = = = = = = = = = = = = = = = = = = = = = = = = = = = = = = = = = =  %
	DepositEther()                           	& Deposits Ethers in the call market contract                         			 \\ \hline
% = = = = = = = = = = = = = = = = = = = = = = = = = = = = = = = = = = = = = = = = = = = = = = = = = = = =  %
	OpenMarket()                             & Opens the market                                                    					 \\ \hline
% = = = = = = = = = = = = = = = = = = = = = = = = = = = = = = = = = = = = = = = = = = = = = = = = = = = =  %
	CloseMarket()                            & Closes the market                                                    					\\ \hline
% = = = = = = = = = = = = = = = = = = = = = = = = = = = = = = = = = = = = = = = = = = = = = = = = = = = =  %
	SubmitBid()                              & Inserts the upcoming bid order messages inside the priority queue    \\ \hline
% = = = = = = = = = = = = = = = = = = = = = = = = = = = = = = = = = = = = = = = = = = = = = = = = = = = =  %
	SubmitAsk()                              & Inserts the upcoming ask order messages inside the priority queue    \\ \hline
% = = = = = = = = = = = = = = = = = = = = = = = = = = = = = = = = = = = = = = = = = = = = = = = = = = = =  %
	MatchOrders()                           & Matches the orders against each other                                			\\ \hline
% = = = = = = = = = = = = = = = = = = = = = = = = = = = = = = = = = = = = = = = = = = = = = = = = = = = =  %
	ClaimTokens()                           & Transfers collateral tokens back to the trader                       			\\ \hline
% = = = = = = = = = = = = = = = = = = = = = = = = = = = = = = = = = = = = = = = = = = = = = = = = = = = =  %
	ClaimEther()                             & Transfers collateral Ethers back to the trader                       			\\ \hline
% = = = = = = = = = = = = = = = = = = = = = = = = = = = = = = = = = = = = = = = = = = = = = = = = = = = =  %
\end{tabular}
\caption{\footnotesize{Primary operations of the call market smart contract.}
\label{tab:callmarket_functions}}
\end{table}

% = = = = = = = = = = = = = = = = = = = = = = = = = = = = = = = = = = = = = = = = = = = = = = = = = = = =  %


To facilitate a safe exchange among buyers and sellers, we implement the Call Market smart contract in the form of a \textit{collateralized}; for market participants to be able to send bid and/or ask orders, they have to first supply tokens (depending on what ERC20 tokens they aim to trade) as collaterals by calling any of the  \texttt{DepositToken()} or \texttt{DepositEther()} methods. The collateralized call market acts as a payment guarantee so traders cannot default on payment or delivery of their assets. 

Followings outline results of the experiments we performed using Javascript and leveraging Mocha testing framework.
 
%============= Worst_Case_Matching_test.js Tables=================== %

 \subsection{Experiments}

Table~\ref{tab:worst_case_matching} outlines the \texttt{Match()} function's computational costs and performance for three different implementations of the call market. We used three different designs of the linked list with mapping priority queue. (1) Every time a trade happens, we use the \texttt{DELETE} method to remove the mapping elements that contains best bid and ask orders from storage completely \textbf{and} update the head and tail pointers of the list. (2) We only update the pointers of the priority queue and do not remove the best bid and ask orders during the entire matching process. Instead we destroy the priority queue smart contract~\footnote{We execute a public function defined in the priority queue smart contract which itself calls the \texttt{self-destruct()} operation and transfers its funds (if any) to the call market contract.} once the matching is completed. (3) In the last implementation, we allow the matching process to be completed and do not clear any on chain data. As it can be seen in Table~\ref{tab:worst_case_matching}, the \texttt{Match()} operation shows a better performance when killing the priority contract and/or not clearing data from the chain. However, leaving unnecessary storage data on chain is not a proper design practice (note that as mentioned in Section~\ref{sec:LLvsLLMapping}, destroying a smart contract does not remove the initial byte code of the contract from the chain but it frees up the state storage and refunds 24000 gas to the caller.)
The last column of Table~\ref{tab:worst_case_matching} reflects the gas cost for matching 1000 pairs of bids and asks for which we set the block gas limit to the maximum of  $2^{53}$ (the Javascript's max safe integer).

Note that this is a \textit{worst case matching} test where all bids and asks are submitted as marketable limit orders with specified prices that would be filled undoubtedly, performed using the current Ethereum gas metrics (block gas limit $=$11,741,495 and 1 gas $=$ 56 gwei)~\footnote{https://ethstats.net/}. 

%Although deleting the priority queue contract adds an extra external call to the \texttt{Match()} function and should technically increase the gas cost, the 24000 gas refunds from \texttt{SELFDESTRUCT} would compensate the cost. So it can still match as same number of orders (before running out of gas) as in the no on chain clearing implementation.


% = = = = = = = = = = = =Worst Case Matching on Linkedlist With Mapping Table = = = = = = = = = = = = =  %

\begin{table}[]
\centering
\begin{tabular}{|c|c|c|c|}
\hline

\textbf{\thead{Linked List with Mapping}}    & \textbf{\thead{Maximum Number \\ of \\ Matched Orders}}      & \textbf{\thead{Net Cost\\in Gas}}          & \textbf{\thead{Net Cost in Gas\\ for \\ 1000 Pairs \\ of Orders}}    \\ \hline

% = = = = = = = = = = = = = = = = = = = = = = = = = = = = = = = = = = = = = = = = = = = = = = = = = = = = = = = = = = = = = = = = = = = = = = = = = = = = = = = = = = = = = = = = = = = =  = = = = = = = = = = = %
	\textbf{\thead{Clearing the Mapping Elements}}         			& 86 pairs                					& 5,433,259                   				& 62,774,170                      \\ \hline
% = = = = = = = = = = = = = = = = = = = = = = = = = = = = = = = = = = = = = = = = = = = = = = = = = = = = = = = = = = = = = = = = = = = = = = = = = = = = = = = = = = = = = = = = = = = =  = = = = = = = = = = = %
	\textbf{\thead{Destroying the Priority Queue Contract}}           	& 210 pairs	                     				& 5,304,211 							& 25,099,713				\\ \hline
% = = = = = = = = = = = = = = = = = = = = = = = = = = = = = = = = = = = = = = = = = = = = = = = = = = = = = = = = = = = = = = = = = = = = = = = = = = = = = = = = = = = = = = = = = = = =  = = = = = = = = = = = %
	\textbf{\thead{No On-chain Clearing}} 		                       	& 210 pairs							& 5,299,877							& 25,095,370				\\ \hline
% = = = = = = = = = = = = = = = = = = = = = = = = = = = = = = = = = = = = = = = = = = = = = = = = = = = = = = = = = = = = = = = = = = = = = = = = = = = = = = = = = = = = = = = = = = = =  = = = = = = = = = = = %

\end{tabular}
\caption{\footnotesize{}
\label{tab:LLMApping_worst_case_matching}}
\end{table}

% = = = = = = = = = = = = = = = = = = = = = = = = = = = = = = = = = = = = = = = = = = = = = = = = = = = =  %


% = = = = = = = = = = = =Worst_Case_Matching Table = = = = = = = = = = = = =  %

\begin{table}[]
\centering
\begin{tabular}{|c|c|c|c|}
\hline

\textbf{\thead{Priority Queue}}    & \textbf{\thead{Maximum Number \\ of \\ Matched Orders}}      & \textbf{\thead{Net Cost\\in Gas}}          & \textbf{\thead{Net Cost in Gas\\ for \\ 1000 Pairs \\ of Orders}} \\ \hline

% = = = = = = = = = = = = = = = = = = = = = = = = = = = = = = = = = = = = = = = = = = = = = = = = = = = = = = = = = = = = = = = = = = = = = = %
	\textbf{\thead{Heap with \\ Dynamic Array}}         				& 38 pairs                & 5,372,679                   	& 457,326,935                      \\ \hline
% = = = = = = = = = = = = = = = = = = = = = = = = = = = = = = = = = = = = = = = = = = = = = = = = = = = = = = = = = = = = = = = = = = = = = = %
	\textbf{\thead{Heap with \\ Static Array}}           				& 42 pairs                & 5,247,636                  	& 333,656,805                       \\ \hline
% = = = = = = = = = = = = = = = = = = = = = = = = = = = = = = = = = = = = = = = = = = = = = = = = = = = = = = = = = = = = = = = = = = = = = = %
	\textbf{\thead{Mapping with \\ keys stored \\ in Heap}} 		& 46 pairs                & 5,285,275                     & 226,499,722                        \\ \hline
% = = = = = = = = = = = = = = = = = = = = = = = = = = = = = = = = = = = = = = = = = = = = = = = = = = = = = = = = = = = = = = = = = = = = = = %
	\textbf{Linkedlist}                       							& 152 pairs                & 5,495,265                     & 35,823,601                        \\ \hline
% = = = = = = = = = = = = = = = = = = = = = = = = = = = = = = = = = = = = = = = = = = = = = = = = = = = = = = = = = = = = = = = = = = = = = = %
	\textbf{\thead{Linkedlist with \\ Mapping}}          				& 86 pairs                 & 5,433,259                     & 62,774,170                        \\ \hline
% = = = = = = = = = = = = = = = = = = = = = = = = = = = = = = = = = = = = = = = = = = = = = = = = = = = = = = = = = = = = = = = = = = = = = = %

\end{tabular}
\caption{\footnotesize{}
\label{tab:worst_case_matching}}
\end{table}


\section{Call Market Different Design and Implementation Details }


% = = = = = = = = = = = =Clearing Mappings = = = = = = = = = = = = =  %

\subsection{Clearing Mappings}

To maintain the collateral balance of each market participant, we use two Solidity type one-to-one mappings that map Ethereum addresses to 256 bits unsigned integers; \texttt{TotalBalance} and \texttt{UnavailableBalance}. Once market closes and orders are matched, the \texttt{UnavailableBalance} needs to be cleared. However, since it is not possible to delete the entire mapping without knowing the keys~\footnote{https://solidity.readthedocs.io/en/v0.5.12/security-considerations.html}, clearing the \texttt{UnavailableBalance} mapping remains a challenging issue to solve. Here we provide a landscape of solutions for that.
%only individual keys and what they map to can be deleted.

\begin{itemize}

%---------+++++++++++++++++++-----------------
\item \textbf{Creating a New Mapping Every Time the Market Opens.} Instead of clearing the \texttt{UnavailableBalance} of traders at the end of a matching process, we could create a new mapping every time the market opens. Note that in this scheme traders can claim their funds back (using the \texttt{ClaimEther()} and/or \texttt{ClaimToken} methods) only when the market is closed (\ie the market is in the \texttt{Closed} state).

%---------+++++++++++++++++++-----------------
\item \textbf{Creating Custom Keys for the Mapping.} We can create custom keys for the mapping by defining a counter as a global variable inside the call market smart contract and increment it only once at the end of the matching process. So instead of clearing the mapping, we only use another portion of it every time the market opens.

%---------+++++++++++++++++++-----------------
\item \textbf{Storing the Mapping in a Data-Contract.} Another design proposal is to create a new smart contract every time the market opens, this data contract only stores the \texttt{UnavailableBalance} mapping and will be destroyed (using the \texttt{SELFDESTRUCT} operation) at the end of the matching process.

%---------+++++++++++++++++++-----------------
\item \textbf{Storing the Mapping keys in an additional Array.} Another common pattern is to create an additional array on top of the mapping and iterate over that. This array (\eg address[]) stores the traders' addresses and enables us to iterate over the mapping and delete individual keys and what they map to at the end of the matching process. Note that this is a gas-costly design pattern as we would need to maintain a secondary data structure. 

%---------+++++++++++++++++++-----------------
\end{itemize}

% = = = = = = = = = = = = = = = = = = = = = = = = = = = = = = = = = = = = = = = = = = = = = = = = = = = = = = = = = = = = = = = = = = = = =  %


% = = = = = = = = = = = =Closing and Reopening the Market = = = = = = = = = = = = =  %

\subsection{Closing and Reopening the Market}

% We use one-shot market for the sake of our performance tests on different data structures. and once we have all the numbers we can think about iterated market.

At the end of the trading period, the market needs to be closed and reopened. However, there is no automatic process to called an Ethereum function and contracts can only run when a function is called. 

%when to close the market?
%What time period?

\begin{itemize}
% = = = = = = = = = = = =How to enforce time on on the Ethereum blockchain? = = = = = = = = = = = = =  %

\item\textbf{{Enforcing Time on the Blockchain.}}

% We want the market to run on a periodic schedule => it is difficult for market participants to call the \texttt{Close()} function periodically.

% However there are incentives for them to do so: (i) they get paid and refunded (financial incentives), and (ii) Closing the market allow market participants to open it again and start their trading activities

% Another solution pattern to enforce the time is to use the Ethereum Alarm Clock 


% = = = = = = = = = = = =Who Pays the Cost for Closing and Reopening the Market?= = = = = = = = = = = = =  %

\item\textbf{{Who Pays the Cost for Closing and Reopening the Market?}}

We think it is useful to explore the landscape of possible designs for closing the market. 

\begin{itemize}

%= = = = = = Miners Close the Market = = = = = = =%
\item \textbf {Miners Close and Reopen the Market.} The difference between the best bid and ask prices is called the \textit{bid-ask spread}. In our design, when the trade occurs between the the highest bid (the highest amount a buyer is willing to pay for an asset) and lowest ask (the lowest amount a seller is willing to accept for an asset) orders, the bid-ask spread is paid to the miner. There are possibilities that (i) no trade occurs or (ii) the bid-ask spread is zero (\ie the best bid and best ask prices are identical). So there is enough economic incentive for the the miner to execute the \texttt{CloseMarket()} function and get refunded as the refund amount could be potentially higher than the bid-ask spread. 
% = = = = = = = = = =Processing Orders in Groups= = = = = = = = = = = = = = = %

\item \textbf{Processing Orders in Groups.} Another solution pattern is to process certain number of bid and ask orders upon every execution of the \texttt{CloseMarket()} function rather than treating them as one substantial transaction. A market participant $P_i$ would process $n$ orders from the previous market (\texttt{CloseMarket(n)}) when sending new orders to the current market. This process continues until all the orders in the previous markets are processed. % incentive? => contribution to the community!?
%= = = = = =  = =  =  Meta txs =  = = = = = = = = =%

\item \textbf {Using the Meta Transactions.} Meta transactions enable users to execute Ethereum functions without paying the gas. Rather than spending gas, users sign their intended action using their private keys and broadcast it to the network with no cost. A third party process (\textit{a relayer}) then crafts the actual transaction on user's behalf, sends the transaction to the Ethereum blockchain, and charges the base contract with the associated fees (see Figure~\ref{fig:meta_tx}). The required gas to pay for the \texttt{Match()} function could be collected as fees. So market participants are charged with certain amounts of fees every time they submit an order, these fees are accumulated in the CallMarket contract and will be used to pay for executing the \texttt{CloseMarket()} function.

% = = = = = = = = = = = =meta_tx Figure = = = = = = = = = = = = =  %

\begin{figure}[htb!p]
\centering
\includegraphics[width=0.8\textwidth]{fig/meta_tx.png}
\caption{\footnotesize{}  \label{fig:meta_tx}}
\end{figure}

% = = = = = = = = = = = = = = = = = = = = = = = = = = = = = = = = = %

%= = = = = =  = =  =  Users close and get refunded with the contract funds (collected as fees during order submissions) =  = = = = = = = = =%

\item \textbf{Using the "Contract Pays" Model.} An alternative solution is to design the market such that the last person to submit an order calls the \texttt{CloseMarket()} function, but in contrast to a normal transaction (where the person initiating the transaction must pay the fee), here the CallMarket contract pays the cost for closing the market and matching the orders respectively. To enforce this design we can use Solidity function modifiers; every time a new order is submitted, a function modifier checks whether (i) the auction period has to end and/or (ii) the maximum number of total orders has reached. If any of these two conditions are met, the \texttt{CloseMarket()} will be called. Again, market participants are charged with certain amounts of fees every time they submit an order, these fees are accumulated in the CallMarket. Once the \texttt{CloseMarket()} is successfully executed and orders are matched, the contract transfers its funds to that person. Note that here the person must still have enough gas to cover the execution of the transaction as the funds will be only transferred after the transaction is fully executed. However, market participants are incentivized to do so as they may receive more ethers than they have spent.

%= = = = = =  = =  =  Rollups=  = = = = = = = = =%

\item \textbf{Using Rollups.} Rollup is a scaling method that moves the storage and computation of the smart contracts off-chain while maintaining the transaction data on the main chain as call-data. In this technique, any Ethereum user can act as a validator; they can execute the \texttt{CloseMarket()}function and only post the new state of the contract (the updated balance of traders) in the form of \textit{assertions} to the main chain. Rollups improve scalability, provide faster and cheaper execution of the contracts, and eliminate the gas limit as the contract is no longer executed on-chain.
In the followings we briefly discuss different rollup proposals and techniques. Each approach uses a different method to ensure correction of assertions:


\begin{itemize}

\item \textbf{{Non-interactive Rollups.}} In this rollup technique, assertions are posted together with a validity proof that would be later used by validators to check if the \texttt{CloseMarket()} function has been executed correctly. ZK-Rollup scheme is one of the solutions that uses ZK-SNARKs to prove the validity of the assertions in zero-knowledge. ZK proofs are quick and cheap to verify but they are expensive and time consuming to generate. These proofs could be generated (i) for free or (ii) the CallMarket contract could collect proportional fees for every trade that is successfully executed.

\item \textbf{{Optimistic Rollups.}} In this scheme, assertions are assumed to be valid if there is no dispute posted about them with a certain window of time (a.k.a. "the challenge period'). Here, dispute resolution is a gas-costly method as the CallMarket contract would have to emulate the transaction on-chain to ensure the correctness the assertion. This scheme introduces a tradeoff between privacy and performance as all the assertions are publicly available and accessible. However, here the new state only reflects the updated balances of traders and no secret is involved. 

\item \textbf {{Multi-round Interactive Rollups.}} In this design paradigm, \textit{pending assertions} are posted on-chain and they are open to dispute. Once the challenge period is over and no challenge is submitted, the assertion is confirmed and the CallMarket contract transitions to the new state (\ie updates traders' balances). This scheme takes the overhead for the CallMarket contract to execute the \texttt{Close()} on-chain by using rounds to the dispute resolutions. The two parties (asserter and challenger) must run an interactive protocol and the CallMarket smart contract would have to act as a referee and decides which party's claim is true. Arbitrum is an example of multi-round interactive rollups that uses an efficient challenge-based protocol to penalize the dishonest parties~\cite{kalodner2018arbitrum}. 

\end{itemize}

%= = = = = =  = =  =  SGX=  = = = = = = = = =%

\item \textbf{{Using Trusted Execution Environments.}} Another way of achieving execution of the \texttt{CloseMarket()} function is incorporating the Ethereum blockchain into the Trusted Execution Environments (TEEs) and decoupling the contract execution from consensus mechanism. TEEs enable secure execution of applications in an isolated processing environment called the \textit{enclave}. Here, the enclave could execute the \texttt{CloseMarket()} function off-chain in TEEs and publish an on-chain attestation Quote to the CallMarket contract. The contract then verifies the correctness of the Quote and if validated correctly, it transitions to the new state. Ekiden is an example that uses Intel SGX to solve the scalability and confidentiality issues with the smart contract execution~\cite{cheng2019ekiden}. A drawback of this scheme is in order to achieve confidentiality-preserving smart contracts we have to trust a trusted party in the form of the hardware manufacturer (\eg Intel).


\end{itemize}
% = = = = = = = = = = = = = = = = = = = = = = = = = = = = = = = = = = = = = = = = = = = = = = = = = = = = = = = = = = = = = = = = = = = = =  %
\end{itemize}

% = = = = = = = = = = = =Transferring the Collateral back to the traders = = = = = = = = = = = = =  %
\subsection{Transferring the Collateral back to Traders.}


% = = = = = = = = = = = = = = = = = = = = = = = = = = = = = = = = = = = = = = = = = = = = = = = = = = = = = = = = = = = = = = = = = = = = =  %



% = = = = = = = = = = = = = = = = = = = = = = = = = = = = = = = = = = = = = = = = = = = = = = = = = = = = = = = = = = = = = = = = = = = = =  %


% = = = = = = = = = = = =Notes Discussed in Meetings = = = = = = = = = = = = =  %

% Our two competitors are Uinswap and Loopring (that uses ZK roll ups):
% 	1- Uniswap: we can argue that in traditional markets we have various traditional exchanges. Each of these exchanges have different trading rules and designs which work for different traders.
%	2- Loopring:
%		2-1: Barrier to entry: In order for a trader to trade specific ERC20 on Loopring, Loopring has to agree to support that token. It might not be difficult and they probably approve the token as long as it is standard compliant.
%		2-2: ZK proofs are quick and cheap to verify but they are expensive and time consuming to generate: (1) now they are generating these proofs for free but later they charge traders for doing that. (2) Looping does not probably generate these proofs per every block so the settlement and 				clearing is not happening per block. So we have faster clearing and settlement as we clear everything per block. 
% 		2-3: Loopring is non-custodial, meaning that the money moves only when the proofs come, so there is a latency in transferring funds to the traders.


% The heap with dynamic array can do 26 pairs of matches at worst case matching whereas Linkedliat does 90 pairs. So if we want to cap the orders over 4 PQs, we’re basically increasing the heap size which makes match more expensive for nothing (every time a match happens and the root is deleted the whole heap is reheapified)

% There will be no ties as we use 2 counters for buy and sell orders. These counters are concatenated them with the price. So we somehow enforce time priority


% = = = = = = = = = = = = = = = = = = = = = = = = = = = = = = = = = = = = = = = = = = = = = = = = = = = = = = = = = = = = = = = = = = = = =  %

% = = = = = = = = = = = = = = = = = = = = = = = = = = = = = = = = = = = = = = = = = = = = = = = = = = = = = = = = = = = = = = = = = = = = =  %
%The tables I changed after estimateGas, Not sure which ones to keep. Should talk to Jeremy 



% = = = = = = = = = = = =Gas costs and refunds for Dequeueing 50 integers from each PQ  = = = = = = = = = = = = =  %

\begin{table}[]
\centering
\begin{tabular}{|c|c|c|c|c|}
\hline

\textbf{\thead{Priority Queue}}    & \textbf{\thead{Net Cost\\in Gas}}      & \textbf{\thead{Total Cost\\in Gas\\(From \texttt{estimateGas})}}      & \textbf{The Difference}    & \textbf{\thead{Gas Refund \\(Manually Calculated)}} \\ \hline

% = = = = = = = = = = = = = = = = = = = = = = = = = = = = = = = = = = = = = = = = = = = = = = = = = = = = = = = = = = = = = = = = = = = = = = %
	\textbf{\thead{Heap with \\ Dynamic Array}}         				& 2,518,131               & 3,283,480		& 765,349             & 750,000                       \\ \hline
% = = = = = = = = = = = = = = = = = = = = = = = = = = = = = = = = = = = = = = = = = = = = = = = = = = = = = = = = = = = = = = = = = = = = = = %
	\textbf{\thead{Heap with \\ Static Array}}           				& 1,385,307                & 2,150,625     	& 765,318             & 750,000                      \\ \hline
% = = = = = = = = = = = = = = = = = = = = = = = = = = = = = = = = = = = = = = = = = = = = = = = = = = = = = = = = = = = = = = = = = = = = = = %
	\textbf{\thead{Mapping with \\ keys stored \\ in Heap}} 		& 2,781,684                & 4,316,229       	& 1,534,545           & 1,500,000                     \\ \hline
% = = = = = = = = = = = = = = = = = = = = = = = = = = = = = = = = = = = = = = = = = = = = = = = = = = = = = = = = = = = = = = = = = = = = = = %
	\textbf{Linked List}                       							& 557,085               	& 1,772,085      	& 1,215,000           & 1,200,000                      \\ \hline
% = = = = = = = = = = = = = = = = = = = = = = = = = = = = = = = = = = = = = = = = = = = = = = = = = = = = = = = = = = = = = = = = = = = = = = %
	\textbf{\thead{Linked List with \\ Mapping}}          				& 731,514              	& 3,731,514       	& 3,000,000     	  &  3,765,000                       \\ \hline
% = = = = = = = = = = = = = = = = = = = = = = = = = = = = = = = = = = = = = = = = = = = = = = = = = = = = = = = = = = = = = = = = = = = = = = %

\end{tabular}
\caption{\footnotesize{The gas metrics associated with dequeuing 50 items from five priority queues.}
\label{tab:PQUnitTests}}
\end{table}
% = = = = = = = = = = = = = = = = = = = = = = = = = = = = = = = = = = = = =  %

% = = = = = = = = = = = = = = = = = = = = = = = = = = = = = = = = = = = = =  %
\begin{table}[]
\begin{tabular}{|c|c|c|c|c|}
\hline
\multirow{2}{*}{Gas Metrics}                        							& \multicolumn{2}{c|}{Linked List}           						& \multicolumn{2}{c|}{Linked List with Mapping}     					\\ \cline{2-5} 
% = = = = = = = = = = = = = = = = = = = = = = = = = = = = = = = = = = = = = = = = = = = = = = = = = = = = = = = = = = = = = = = = = = = = = =  = = = = = =  = = = = = = = = = = = = = = = =  = = = = = =  = = = = = = %
                                                                						& with \texttt{SELFDESTRUCT} 		& without \texttt{SELFDESTRUCT}                	& with \texttt{delete} 		& without \texttt{delete}              		\\ \hline                                                                                                                                                                                                                                                          % = = = = = = = = = = = = = = = = = = = = = = = = = = = = = = = = = = = = = = = = = = = = = = = = = = = = = = = = = = = = = = = = = = = = = =  = = = = = =  = = = = = = = = = = = = = = = =  = = = = = =  = = = = = = %
                   Net Cost in Gas                          					& 557,085                    			& 721,370                     				& 731,514                     	& 334,689                      				\\ \hline
% = = = = = = = = = = = = = = = = = = = = = = = = = = = = = = = = = = = = = = = = = = = = = = = = = = = = = = = = = = = = = = = = = = = = = =  = = = = = =  = = = = = = = = = = = = = = = =  = = = = = =  = = = = = = %
\thead{Total Cost in Gas \\ (from \texttt{estimateGas})} 		& 1,772,085                  			& 736,370                  					& 3,731,514                     	& 1,099,689                       				\\ \hline
% = = = = = = = = = = = = = = = = = = = = = = = = = = = = = = = = = = = = = = = = = = = = = = = = = = = = = = = = = = = = = = = = = = = = = =  = = = = = =  = = = = = = = = = = = = = = = =  = = = = = =  = = = = = = %
		    The Difference                                             		& 1,215,000                  			& 15,000                      				& 3,000,000                    	& 765,000                       				\\ \hline
% = = = = = = = = = = = = = = = = = = = = = = = = = = = = = = = = = = = = = = = = = = = = = = = = = = = = = = = = = = = = = = = = = = = = = =  = = = = = =  = = = = = = = = = = = = = = = =  = = = = = =  = = = = = = %

\end{tabular}
\caption{\footnotesize{}
\label{tab:LLvsLLMapping}}
\end{table}
% = = = = = = = = = = = = = = = = = = = = = = = = = = = = = = = = = = = = =  %




%======================================== %

% = = = = = = = = = = = = = = = = = = = = = = = = = = = = = = = = = = = = = = = = = =

\section{Concluding Remarks}


 
%\subsubsection*{Acknowledgements.} J. Clark thanks ...

































