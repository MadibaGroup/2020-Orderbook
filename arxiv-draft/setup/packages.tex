% !TEX root = ../main.tex


%------------------------Packages----------------------%

% = = = Full page for the arxiv version
\usepackage{fullpage}
% = = = 

\usepackage{balance} %for balanced columns on the last page
% = = = Graphics
%\usepackage[pdftex]{graphicx}
%\graphicspath{{figures/}}
%\DeclareGraphicsExtensions{.jpg,.png}
\usepackage{floatrow}

% = = = Subfig (note not subfigure)
\usepackage[caption=false,font=footnotesize]{subfig}
% Table float box with bottom caption, box width adjusted to content
\newfloatcommand{capbtabbox}{table}[][\FBwidth]

% = = = Math Symbols
\usepackage{amsmath}
%\usepackage{amstext,amssymb,amsthm}
\usepackage{bbm}
\usepackage{stmaryrd}

% = = = Other
\usepackage{array}
\usepackage[hyphens]{url}
\usepackage[pdftitle=Title,pdfauthor=Anonymous]{hyperref}
\usepackage{ wasysym }
\usepackage{adjustbox}
\usepackage{booktabs}
\usepackage{multirow}
\usepackage{rotating}
\usepackage{makecell}
\usepackage{hhline}
\usepackage{lscape}
\usepackage{tabu}
\usepackage{tikz}
\usepackage{tikzsymbols}
\usepackage{textcomp} %Using tilde
\usetikzlibrary{shapes,backgrounds} %Venn diagram
\usepackage[justification=centering]{caption} % to add caption in tabular environment using the command \captionof{table}{Your caption here} 
\usepackage{colortbl}




\usepackage{blindtext}
\usepackage[utf8]{inputenc}
\usepackage[T1]{fontenc}
\usepackage{amsmath}
\usepackage{amsfonts}
%\usepackage{amssymb}
\usepackage{tabularx}
\usepackage{blindtext}
\usepackage{booktabs}
\usepackage{makecell}

\usepackage{cellspace}

\usepackage{longtable}





\usepackage{diagbox} 


%---------------------------------------------------%
\usepackage{listings, xcolor}
\renewcommand{\lstlistingname}{Code}
\definecolor{verylightgray}{rgb}{.97,.97,.97}
\lstdefinelanguage{Solidity}{
keywords=[1]{anonymous, assembly, assert, balance, break, call, callcode, case, catch, class, constant, continue, contract, debugger, default, delegatecall, delete, do, else, event, export, external, false, finally, for, function, gas, if, implements, import, in, indexed, instanceof, interface, internal, is, length, library, log0, log1, log2, log3, log4, memory, modifier, new, payable, pragma, private, protected, public, pure, push, require, return, returns, revert, selfdestruct, send, storage, struct, suicide, super, switch, then, this, throw, transfer, true, try, typeof, using, value, view, while, with, addmod, ecrecover, keccak256, mulmod, ripemd160, sha256, sha3}, % generic keywords including crypto operations
	keywordstyle=[1]\color{blue}\bfseries,
	keywords=[2]{Stages,States, address, bool, byte, bytes, bytes1, bytes2, bytes3, bytes4, bytes5, bytes6, bytes7, bytes8, bytes9, bytes10, bytes11, bytes12, bytes13, bytes14, bytes15, bytes16, bytes17, bytes18, bytes19, bytes20, bytes21, bytes22, bytes23, bytes24, bytes25, bytes26, bytes27, bytes28, bytes29, bytes30, bytes31, bytes32, enum, int, int8, int16, int24, int32, int40, int48, int56, int64, int72, int80, int88, int96, int104, int112, int120, int128, int136, int144, int152, int160, int168, int176, int184, int192, int200, int208, int216, int224, int232, int240, int248, int256, mapping, string, uint, uint8, uint16, uint24, uint32, uint40, uint48, uint56, uint64, uint72, uint80, uint88, uint96, uint104, uint112, uint120, uint128, uint136, uint144, uint152, uint160, uint168, uint176, uint184, uint192, uint200, uint208, uint216, uint224, uint232, uint240, uint248, uint256, var, void, ether, finney, szabo, wei, days, hours, minutes, seconds, weeks, years},	% types; money and time units
	keywordstyle=[2]\color{teal}\bfseries,
	keywords=[3]{block, blockhash, coinbase, difficulty, gaslimit, number, timestamp, msg, data, gas, sender, sig, value, now, tx, gasprice, origin},	% environment variables
	keywordstyle=[3]\color{violet}\bfseries,
	identifierstyle=\color{black},
	sensitive=false,
	comment=[l]{//},
	morecomment=[s]{/*}{*/},
	commentstyle=\color{gray}\ttfamily,
	stringstyle=\color{red}\ttfamily,
	morestring=[b]',
	morestring=[b]"
}

\lstset{
	language=Solidity,
	backgroundcolor=\color{verylightgray},
	extendedchars=true,
	basicstyle=\footnotesize\ttfamily,
	showstringspaces=false,
	showspaces=false,
	numbers=left,
	numberstyle=\footnotesize,
	numbersep=9pt,
	tabsize=2,
	breaklines=true,
	showtabs=false,
	captionpos=b,
      escapeinside={<@}{@>},
}

%\setlength\doublerulesep{3pt}











%------------------------END----------------------%  
  

