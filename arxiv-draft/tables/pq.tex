% !TEX root = ../main.tex

% = = = = = = = = = = = =Operations for a generic Priority Queue Table = = =  = = = = = = = = = = = %
\begin{table}[t]
%\setlength{\tabcolsep}{1pt}% Shrink \tabcolsep
\centering
\scriptsize
\begin{tabular}{|c|c|}
\hline

\textbf{Operation}   & \textbf{Description}    \\ \hline

% = = = = = = = = = = = = = = = = = = = = = = = = = = = = = = = = = = = = = = = = = = = = = = = = %
	\textbf{Enqueue()}       	& Inserts an element into the priority queue                        \\ \hline
	\textbf{Dequeue()}		& Removes and returns the highest priority element 		\\ \hline
	\textbf{isEmpty()}			& Checks if the priority queue is empty 					\\ \hline
% = = = = = = = = = = = = = = = = = = = = = = = = = = = = = = = = = = = = = = = = = = = = = = = =  %

\end{tabular}
\caption{\footnotesize{Operations for a generic priority queue}.
\label{tab:PQ_API}}
\end{table}
% = = = = = = = = = = = = = = = = = = = = =  = = = = = = = = = = =  = = = = = = = = = == = = =  = =%